\chapter{Démarche}
%---------------------------------------------------------------------------------------------------------------
\section{Liste des films}

Tout d'abord, nous avons trouvé une liste de films qui provient du site \url{IMDb.com}. Elle contient des informations sur 3393 films connus. Cela nous permet d'avoir une liste fixe et de ne pas avoir besoin d'aller les chercher à chaque fois sur internet. Nous avons donc stocké ce fichier que nous utiliserons pas la suite. Ce fichier est formaté de la manière suivante : une ligne correspond à un film, chaque ligne contient 5 informations sur le film :

\begin{itemize}
 \item Le titre du film
 \item L'année de sortie du film
 \item La note qu'à le film sur IMDb
 \item Le nombre de vote qu'à reçu le film sur IMDb
 \item Le genre (plusieurs genres séparés par des virgules)
\end{itemize}

Chacune des ces cinq informations est séparée par deux espaces. Voici un aperçu de cette structure de fichier : \\

\begin{lstlisting}[language=bash]
  The Pianist  2002  8.5  151864  Biography,Drama,War
  The Godfather  1972  9.2  450433  Crime,Drama,Thriller
  Casper  1995  5.7  28316  Comedy,Family,Fantasy
  Hostel  2005  5.7  69499  Mystery,Thriller
  Juno  2007  7.8  163142  Comedy,Drama,Romance
\end{lstlisting}

Cette structure nous permet donc d'extraire facilement les informations voulues avec la fonction Python suivante : \\

\begin{lstlisting}[language=python]
  def extractMovieList(fname):

      with open(fname) as f:
          content = f.readlines()
    
      title = []
      year = []
      grade = []
      votes = []
      genre = []

      for line in content:
          infos = line.split('  ')
          title.append(infos[0].replace('"',''))
          year.append(infos[1])
          grade.append(infos[2])
          votes.append(infos[3])
          genre.append(infos[4])

      return [title, year, grade, genre]
\end{lstlisting}

%---------------------------------------------------------------------------------------------------------------
\section{Recherche d'information sur un film}

Les données disponibles dans le fichier \texttt{movielist.txt} ne nous permetteront pas de classifier les films. C'est juste une liste qui nous permet d'avoir une quantité de titre de films ainsi que les genres de chaque film. Nous voulons alors rechercher les descriptifs de chaque films. Pour cela, nous utilisons une API\footnote{http://fr.wikipedia.org/wiki/Interface\_de\_programmation} d'IMDb qui est documentée à l'adresse \url{http://omdbapi.com}. Cette API nous permet de chercher des informations sur un film soit en donnant le titre, soit en donnant l'identitiant IMDb du film (ce que nous n'avons pas). 

Nous allons alors prendre la liste de films et faire une requête sur cette API pour chaque titre de film que nous avons. De temps en temps, aucun résultat ne sort, nous passons simplement au suivant. Nous enregistrons alors tous les résultats dans un fichier sous format JSON.\footnote{http://fr.wikipedia.org/wiki/JavaScript\_Object\_Notation} \\

\begin{lstlisting}[language=python]
  def getMovieDescriptions(movieTitles, fname):

      movieInfos = []
      base_url = "http://www.omdbapi.com/?plot=full&r=json&t="
    
      for title in movieTitles:
          print 'fetching ' + title
          url = base_url + title
          r = requests.get(url)
          movieInfos.append(r.json())

      with open(fname, 'wb') as outfile:
          json.dump(movieInfos, outfile)
\end{lstlisting}

Pour chaque film, nous recevons différentes informations, dont un lien vers une image, le titre, la durée du film, l'année, les acteurs, le pays, la langue, l'identifiant IMDb, un descriptif, etc. Ce qui nous intéresse à première vue est le descriptif. Seulement après avoir analysé ces descriptifs, nous remarquons qu'ils sont souvent courts, ils comportent pour la plupart moins de cinquante mots. Cela est beaucoup trop peu pour pouvoir trouver des ressemblances entres les films et les classifiers. 

Nous avons alors eu l'idée d'utiliser l'identifiant IMDb de chaque films pour aller chercher plus d'informations sur les films. Après quelques recherche, nous trouvons qu'il y a une page spécifique pour chaque film sur le site IMDb où se trouve un résumé plus détaillé du film. L'adresse est formée de la façon suivante : \textit{'http://www.imdb.com/title/[imdbId]/synopsis'}. Il nous reste donc à remplacer l'identifiant que l'on veut dans cette URL et nous arrivons sur un beau descriptif, qu'il faut tout de même extraire de la page web (parser du HTML est toujours une partie de plaisir, quoique).

Un script Python va nous permettre d'aller chercher le descriptif de chaque film et de l'enregistrer cette fois dans un fichier séparé pour chaque film. Voici ce script : \\

\begin{lstlisting}[language=python]
  import requests
  import json
  import BeautifulSoup
  import re
  import os.path

  def extractFromFile(fname):
      file = open(fname, 'r')
      array =  file.read()
      data  = json.loads(array)
    
      # remove element that doesn't have the Title or imdbID fields
      data = [item for item in data if 'Title' in item and 'imdbID' in item]

      return data

  def extractSynopsis(imdbId):
      url = 'http://www.imdb.com/title/' + imdbId + '/synopsis'
      r = requests.get(url)

      soup = BeautifulSoup.BeautifulSoup(r.text)
      div = soup.find("div", {"id": "swiki.2.1"})
      if div :
          synopsis = re.sub('<[^<]+?>', '', div.text)
          if synopsis:
              return synopsis

      return None

  if __name__ == '__main__':
    
      # n --> 1, 3, 5, 10, 50, 100, 3393
      n = 3393

      fname = 'data2/moviedescriptions' + str(n) + '.json'
      movies_data = extractFromFile(fname)

      i = 0
      emptyQty = 0
      for movie in movies_data:
          i = i + 1
          imdbId = movie['imdbID']
          fname = 'data2/synopsis/' + imdbId + '.txt'
          if not os.path.isfile(fname):
              print str(i) + '/' + str(n) + ' : ' + imdbId
              synopsis = extractSynopsis(imdbId)
              if synopsis:
                  f = open(fname,"w")
                  f.write(synopsis.encode('utf8'))
                  f.close()
              else:
                  emptyQty = emptyQty + 1
                  print imdbId + ' synopsis is empty (' + str(emptyQty) + ')'

      print 'Total empty synopsis : ' + str(emptyQty)
      print 'Total synopsis : ' + str(n - emptyQty)
\end{lstlisting}

Tous les films n'avaient pas leur page de descriptif correspondante. Sur les 3393 films initiaux, seuls 2542 ont pu être extrait et enregistré. C'est tout de même assez pour faire nos tests. Nous continuons donc avec cela.

%---------------------------------------------------------------------------------------------------------------
\section{Création du dataset}

Nous avons maintenant tout en main, et en local, pour pouvoir générer les structures de données spécifiques. Nous avons besoin de trois structures différentes :

\begin{itemize}
 \item Un vecteur contenant la liste des films
 \item Un vecteur contenant la liste des mots qui apparaîssent
\end{itemize}




%---------------------------------------------------------------------------------------------------------------
\section{Création des vecteurs des documents}

asdf


%---------------------------------------------------------------------------------------------------------------
\section{Amélioration des vecteur de documents}

tfidf, suppression mot unique



%---------------------------------------------------------------------------------------------------------------
\section{Matrice de distances}

\label{matrice-distance}
Pour avoir une notion de distance entre les documents nous avons créé une matrice des distances. Pour faire cela, on a utilisé la librairie "hcluster" \footnote{Hcluster, \url{https://code.google.com/p/scipy-cluster/}} qui prend en entrée une matrice de vecteurs des documents et elle permet de définir une métrique de distance : dans notre cas on a utilisé la fonction "cosine".

Le code pour la génération de la matrice de distance est le suivant : \\

\begin{lstlisting}[language=python]
  titles, words, matrix = extractArrays(infos)
  distanceMatrix =pairwise_distances(matrix, metric='cosine')
\end{lstlisting}


%---------------------------------------------------------------------------------------------------------------
\section{Liste des films similaires}

Pour voir si la matrice des vecteurs des documents décrit bien les documents nous avons implémenté une fonction qui, avec un film en paramètre, retourne les N films las plus similaires à celui-ci.

La fonction prend en paramètre l'id du film initial, le nombre des films similaires à afficher, la matrice de distance décrite dans la section \ref{matrice-distance} ainsi que le vecteur des titres des films.

Le code de cette fonction est le suivant : \\

\begin{lstlisting}[language=python]
def printClosest(idxFilm, numclosest, distanceMatrix, titles):
  print titles[idxFilm]+":"
  cloasest= heapq.nsmallest(numclosest,range(len(distanceMatrix[idxFilm])),distanceMatrix[idxFilm].take)
  for idx, val in enumerate(cloasest):
      print  "\t"+str(idx)+" "+titles[val]
\end{lstlisting}


%---------------------------------------------------------------------------------------------------------------
\section{Clustering hiérarchique}

Une fois avoir évalué quelques films avec la liste des films similaires on a voulu vérifier mieux cela en faisan un clastering hiérarchique en vérifiant que les filmes regroupé dans un clastering sont une catégorie similaire.

Pour faire cela, on utilise la libraire "hcluster" pour effectuer le clustering hiérarchique en se basant sur la matrice de distances.

Le clustering hiérarchique sera visualisé dans un dendrogram représentant les regroupements des films. \\

\begin{lstlisting}[language=python]
Z=linkage(distanceMatrix,method='average')#,method='centroid')
print Z.shape
image=dendrogram(Z,labels=titlesCat, distance_sort='descendent',
         leaf_font_size=2, orientation='left', show_contracted=False)
pylab.savefig("images/clustering100_tf_idf.png",dpi=300,bbox_inches='tight')	 	  
\end{lstlisting}

Pour nous aider à la visualisation du graphique on a utilisé une fonction utilitaire pour visionner les films les plus similaires que sont regroupé par l'algorithme de clustering.

Pour faire cela on a fait la fonction suivante : \\

\begin{lstlisting}[language=python]
print "first closest cluster"
for idx in range(10):
    lenTitle=len(titles)
    if (int(Z[idx,0])<lenTitle) & (int(Z[idx,1])<lenTitle):
        print "itr "+str(idx)+":\n"+titlesCat[int(Z[idx,0])]+" "+titlesCat[int(Z[idx,1])]
\end{lstlisting}

La variable Z a été calculée par la fonction de la libraire "hcluter" et on peut voir, à chaque itération, quel film a été mis ensemble avec quelle outre.	


%---------------------------------------------------------------------------------------------------------------
\section{Map de kohonen}

Une fois avoir vu les résultats depuis le clustering hiérarchique on peut analyser la matrice initiale avec un algorithme un peu plus avancé et qui donne des résultats plus visuels.

On a utilisé l'algorithme de Khonen qui prend en entrée notre matrice des vecteurs des documents et produise en sortie une map en couleur avec la position de chaque filme par rapport a les autres. Les couleurs de la map représente la distance entre chaque film. 

Pour faire cela, on a utilisé la base du code fait dans le TP 
4 \footnote{TP Kohonen, HES-SO, \url{http://193.134.218.37/labs/lab4/lab4_assignment.html}}. Dans celui-là  on a modifié la métrique de mesure de la distance et la construction de la matrice initiale : \\

\begin{lstlisting}[language=python]
# define cosine metric for configure distance metrix on kohonen
def cosine_metric(x, y):
	#Returns the cosine distance between x and y.
	nx = np.sqrt(np.sum(x * x, axis=-1))
	ny = np.sqrt(np.sum(y * y, axis=-1))
	# the cosine metric returns 1 when the args are equal, 0 when they are
	# orthogonal, and -1 when they are opposite. we want the opposite effect,
	# and we want to make sure the results are always nonnegative.
	return 1 - np.sum(x * y, axis=-1) / nx / ny

params = kohonen.Parameters(dimension=len(words), shape=(side,side*2), metric=cosine_metric)
kmap = kohonen.Map(params)
\end{lstlisting}




%---------------------------------------------------------------------------------------------------------------
% Code integration example
%\begin{lstlisting}[language=bash]
%  sudo apt-get update
%  sudo apt-get install drupal7
%\end{lstlisting}

% Image integration example
%\begin{figure}[h]
%  \centering
%    \includegraphics[width=1\linewidth]{img/drupalFirstPage.png}
%  \caption{Page d'accueil du site créé avec Drupal sur une instance EC2}
%  \label{drupalfirstpage}
%\end{figure}

% Image side-by-side
%\begin{figure}[h!]
%    \centering
%    \begin{tabular}{cccc}
%      \includegraphics[width=.14\linewidth]{randomTree_n5.png} &
%      \includegraphics[width=.22\linewidth]{randomTree_n10.png} &
%      \includegraphics[width=.22\linewidth]{randomTree_n15.png} \\
%      (a) & (b) & (c)\\
%    \end{tabular}
%    \caption{Arbres aléatoires où (a) n=5 (b) n=10 (c) n=15
%    \label{randomTrees}}
%\end{figure}