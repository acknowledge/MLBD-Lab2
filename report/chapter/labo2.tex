\newpage
\section*{Évaluation du modèle}

Dans cette partie nous avons testé l'algorithme avec les meilleurs paramètres trouvés auparavant et nous l'avons testé avec trois combinaisons de poids pour la fonction de fitness (table \ref{table_param}) :


\begin{table}[h]
  \centering
  \begin{tabular}{|l|l|l|l|}
  \hline
  \textbf{Criterion weight} & \textbf{Exp1} & \textbf{Exp2} & \textbf{Exp3} \\ \hline
  accuracy                 & 1             & 1             & 1             \\ \hline
  Sensivity                & 0             & 1             & 1             \\ \hline
  Specificity              & 0             & 0             & 1             \\ \hline
  RMSE                     & 0             & 0             & 0.4           \\ \hline
  Size                     & 0.1           & 0.1           & 0.1           \\ \hline
  Fitness max              & 0.732         & 0.81          & 0.748         \\ \hline
  Average                  & 0.579         & 0.65          & 0.58          \\ \hline
  accuracy                 & 0.734         & 0.725         & 0.75          \\ \hline
  Sensivity                & 0.657         & 0.908         & 0.676         \\ \hline
  Specificity              & 0.8           & 0.571         & 0.820         \\ \hline
  \end{tabular}
  \caption{\label{table_param} Table de résultats}
\end{table}

On peut voir du tableau ci-dessus que la valeur maximale du "fitness" est arrivée dans l'expérience 2 où on donne un poids assez grand au \textbf{taux d'erreur (accuracy) et à la sensivité} et un poids plus petit à la spécificité.


En plus de la valeur de fitness, on peut aussi voir les règles générées par le système. Les règles du système de l'expérience 2 sont les suivantes:
\\
\begin{enumerate}
  \item IF V5 is MF 1 AND V4 is MF 0 THEN Out is MF 1
  \item IF V5 is MF 1 AND V4 is MF 0 THEN Out is MF 1
  \item IF V15 is MF 1 AND V13 is MF 1 THEN Out is MF 1
  \item IF V9 is MF 1 AND V2 is MF 0 AND V4 is MF 0 AND V8 is MF 1 THEN Out is MF 1
  \item IF V9 is MF 1 AND V2 is MF 0 AND V4 is MF 0 AND V8 is MF 1 THEN Out is MF 1
  \item IF V8 is MF 0 AND V9 is MF 1 THEN Out is MF 0
  \item IF V8 is MF 0 AND V9 is MF 1 THEN Out is MF 0
  \item IF V10 is MF 1 AND V11 is MF 0 AND V15 is MF 0 AND V13 is MF 1 THEN Out is MF 1 ELSE : Out is 1  
\end{enumerate}
\\\\
Chaque variable VX correspond à une caractéristique utilisée pour générer le modèle. Ces caractéristiques sont expliquées sur la page \url{http://archive.ics.uci.edu/ml/datasets/Arrhythmia}.
Si on prend par exemple la règle 1, on peut la décrire de la façon suivante : si V5 (QRS) est "grand" et V4 (Weight) est petit, alors le patient a de bonnes chances d'avoir des troubles d'arythmie cardiaque (sortie MF 1).

On peut voir que les règles 1-2,3,6-7 sont également simple et facilement compréhensible. 
Les règles 6-7 expliquent, par exemple, que si le "T-interval" (V8) est petit et le P-interval (V9) est grand, alors le patient a de bonnes chances de n'avoir par de troubles d' arythmie cardiaque.


\section*{Un plus petit "Fuzzy system"}
Âpres avoir évalué notre modèle avec différentes fonctions de fitness on veut maintenant générer un modèle plus petit en changeant encore une fois l'évaluation de la fonction de fitness ainsi que changent les nombres des règles et les nombres des variables utilisés.

On a donc appliqué cette modification et on a reporté les résultats dans le tableau suivant (table \ref{table4}):




 
 
\begin{table}[h]
\centering

\label{my-label}
\begin{tabular}{|l|l|l|l|l|}
\hline
\textbf{Criterion weigth} & \textbf{Exp4} & \textbf{Exp5} & \textbf{Exp6} & \textbf{Exp7} \\ \hline
Accuracy                 & 1             & 1             & 1             & 1             \\ \hline
Sensivity                & 1             & 1             & 1             & 1             \\ \hline
Specificity              & 0             & 0             & 0             & 0             \\ \hline
RMSE                     & 0             & 0             & 0             & 0             \\ \hline
Size                     & 0.9           & 0.1           & 0.9           & 1             \\ \hline
\textbf{Fitness max}     & \textbf{0.72} & \textbf{0.65} & \textbf{0.80} & \textbf{0.80} \\ \hline
Average fitness          & 0.64          & 0.569         & 0.67          & 0.73          \\ \hline
Accuracy                 & 0.47          & 0.67          & 0.47          & 0.47          \\ \hline
Sensivity                & 0.995         & 0.67          & 0.99          & 0.995         \\ \hline
Specificity              & 0.02          & 0.67          & 0.028         & 0.02          \\ \hline
Size                     & 1             & 0.5           & 1             & 1             \\ \hline
Régle - variable/règle  & 5 règles - 2 règles      &        &               &               \\ \hline
\end{tabular}
\caption{\label{table4}Table des résultats}
\end{table}

Les résultats du tableau \ref{table4}  montre 4 différentes expériences qu’ on a faites en changeante le poids de la grandeur du modèle et le nombre de variables des règles générées. Dans l'expérience "Exp4" on a mis un poids sur la grandeur dans l'évaluation de la fonction de fitness de 0.9. Depuis cette expérience on obtient une valeur de fitness maximale de  \textbf{0.72} et une grandeur du modèle de \textbf{1}. Dans l'expérience "Exp5" on a changé le nombre de règles à \textbf{2} et les nombres des  variables utilisées pour la description des règles générées par le système à \textbf{2}. Les résultats montre que le système à un taille de \textbf{0.5} mais un fitness de \textbf{0.65}.
Dans les expériences suivantes (Exp6-7) on a combiné les deux paramètres pour essayer d'avoire de performance meilleure. Effectivement la valeur de fitness est plus haute si on combine ces deux modifications.





 
 
 
 

